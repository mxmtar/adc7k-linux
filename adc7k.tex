\documentclass[a4paper]{article}
\usepackage[T1,T2A]{fontenc}
\usepackage[utf8]{inputenc}
\usepackage[english,russian]{babel}
\righthyphenmin=2

\title{Модули ядра Linux для поддержки плат ADC7K\\adc7k-linux-1.0}
\author{Максим Тарасевич}
\date{Polygator --- 2016}

\begin{document}


\maketitle

\newpage
\tableofcontents

\newpage
\section{Введение}

Данный документ содержит информацию о сборке, установке и работе с
модулями для поддержки плат ADC7K в операционных системах на базе ядра Linux.

\section{Сборка}

Перед сборкой модулей необходимо установить утилиты make, gcc и заголовочные
файлы ядра linux.
В дистрибутивах на базе RedHat
\begin{small}\begin{verbatim}
adc7k@adc7k ~/adc7k-linux $ yum install make gcc kernel-devel
\end{verbatim}\end{small}
или
\begin{small}\begin{verbatim}
adc7k@adc7k ~/adc7k-linux $ dnf install make gcc kernel-devel
\end{verbatim}\end{small}
В дистрибутивах на базе Debian
\begin{small}\begin{verbatim}
adc7k@adc7k ~/adc7k-linux $ apt-get install make gcc linux-headers-`uname -r`
\end{verbatim}\end{small}

Далее в рабочей директории размещаем архивный файл \mbox{\texttt{adc7k-linux-1.0.tar.gz}}
с исходными кодами модулей и распаковываем его
\begin{small}\begin{verbatim}
adc7k@adc7k ~ $ tar xzf adc7k-linux-1.0.tar.gz
\end{verbatim}\end{small}
и переходим в директорию
\begin{small}\begin{verbatim}
adc7k@adc7k ~ $ cd adc7k-linux-1.0
\end{verbatim}\end{small}
которая содержит следующие файлы
\begin{small}\begin{verbatim}
adc7k@adc7k ~/adc7k-linux-1.0 $ ls -l
total 100
-rw-r--r-- 1 adc7k adc7k  1253 Sep 12 10:21 Makefile
drwxr-xr-x 2 adc7k adc7k  4096 Sep 12 10:21 adc7k
-rw-r--r-- 1 adc7k adc7k 14817 Sep 12 10:21 adc7k-base.c
-rw-r--r-- 1 adc7k adc7k 39068 Sep 12 10:21 adc7k-cpci3u-base.c
-rw-r--r-- 1 adc7k adc7k 17371 Sep 12 10:21 adc7k-pseudo-base.c
-rw-r--r-- 1 adc7k adc7k    75 Sep 12 10:21 adc7k-udev.rules
drwxr-xr-x 2 adc7k adc7k  4096 Sep 12 10:21 scripts
adc7k@adc7k ~/adc7k-linux-1.0 $
\end{verbatim}\end{small}

Сборку модулей производим выполненем команды \texttt{make} в текущей директории.
Здесь мы можем убедится в наличии всех инструментов и заголовочных файлов ядра
linux текущей версии необходмых для сборки. В следующем примере показан вывод
команды в случае успешной сборки.
\begin{small}\begin{verbatim}
adc7k@adc7k ~/adc7k-linux $ make
make[1]: Entering directory '/usr/src/linux-headers-3.16.0-4-586'
  CC [M]  /home/adc7k/adc7k-linux/adc7k-base.o
  LD [M]  /home/adc7k/adc7k-linux/adc7k.o
  CC [M]  /home/adc7k/adc7k-linux/adc7k-cpci3u-base.o
  LD [M]  /home/adc7k/adc7k-linux/adc7k-cpci3u.o
  CC [M]  /home/adc7k/adc7k-linux/adc7k-pseudo-base.o
  LD [M]  /home/adc7k/adc7k-linux/adc7k-pseudo.o
  Building modules, stage 2.
  MODPOST 3 modules
  CC      /home/adc7k/adc7k-linux/adc7k-cpci3u.mod.o
  LD [M]  /home/adc7k/adc7k-linux/adc7k-cpci3u.ko
  CC      /home/adc7k/adc7k-linux/adc7k-pseudo.mod.o
  LD [M]  /home/adc7k/adc7k-linux/adc7k-pseudo.ko
  CC      /home/adc7k/adc7k-linux/adc7k.mod.o
  LD [M]  /home/adc7k/adc7k-linux/adc7k.ko
make[1]: Leaving directory '/usr/src/linux-headers-3.16.0-4-586'
adc7k@adc7k ~/adc7k-linux $
\end{verbatim}\end{small}

\section{Установка}
\subsection{Установка модулей}

Для установки модулей применяем в текущей директории команду
\mbox{\texttt{make install}}. Для выполнения данной команды необходимы права
суперпользователя. В примере используется команда \texttt{sudo}.
\begin{small}\begin{verbatim}
adc7k@adc7k ~/adc7k-linux $ sudo make install
make[1]: Entering directory `/usr/src/linux-headers-3.16.0-4-586'
  INSTALL /home/adc7k/adc7k-linux/adc7k-cpci3u.ko
  INSTALL /home/adc7k/adc7k-linux/adc7k-pseudo.ko
  INSTALL /home/adc7k/adc7k-linux/adc7k.ko
  DEPMOD  3.16.0-4-586
make[1]: Leaving directory '/usr/src/linux-headers-3.16.0-4-586'
install -m 755 -d "/usr/include/adc7k"
for header in adc7k/*.h ; do \
	install -m 644 $header "/usr/include/adc7k" ; \
done
adc7k@adc7k ~/adc7k-linux $
\end{verbatim}\end{small}

В результате мы видим, что модули находятся в директории библиотек, которые
соответствуют версии ядра linux, для которой мы производили сборку.
\begin{small}\begin{verbatim}
adc7k@adc7k ~/adc7k-linux $ ls -l /lib/modules/`uname -r`/adc7k
total 60
-rw-r--r-- 1 root root 27024 Sep 12 10:28 adc7k-cpci3u.ko
-rw-r--r-- 1 root root 14904 Sep 12 10:28 adc7k-pseudo.ko
-rw-r--r-- 1 root root 14892 Sep 12 10:28 adc7k.ko
adc7k@adc7k ~/adc7k-linux $
\end{verbatim}\end{small}

Файл модуля \mbox{\texttt{adc7k.ko}} является базовым и предназначен для
регистрации подсистемы \texttt{adc7k} в ядре linux. Модуль \mbox{\texttt{adc7k-cpci3u.ko}}
предназначен для обслуживания плат ADC7K \mbox{CompactPCI-3U} и зависит от \mbox{\texttt{adc7k.ko}}.
В модуле \mbox{\texttt{adc7k-pseudo.ko}} реализовано псевдоустройство, которое
предназначено для отработки программного обеспечения с одинаковым интерфейсом, на
уровне апаратной абстракции, как и у плат ADC7K \mbox{CompactPCI-3U}. Это может быть удобным в случае
невозможности установки плат ADC7K на машине разработчика функционального программного обеспечения.
Модуль \mbox{\texttt{adc7k-pseudo.ko}} также зависит от \mbox{\texttt{adc7k.ko}}.
Немного выше мы могли видеть, выполняя команду установки модулей, операцию
\mbox{\texttt{DEPMOD  3.16.0-4-586}}. Эта операция формирует карту зависимостей между
модулями, а также, в часности для устройств PCI, сопоставляет идентификатору устройства модуль
который его обслуживает. Это дает возможность системе автоматически подгружать зависимые модули как при
ручной загрузке модуля, так и автоматически при запуске операционной системы.

К сожалению не все скрипты установки, показывая операцию {\texttt{DEPMOD}}, её на самом деле проводят.
Поэтому данную операцию необходимо произвести вручную. Она также требует права суперпользователя.
\begin{small}\begin{verbatim}
adc7k@adc7k ~/adc7k-linux $ sudo depmod
\end{verbatim}\end{small}

На данном этапе модули установлены и готовы к загрузке.

\subsection{Расширение прав доступа к файлам устройств}

Необходимо заметить, что по умолчанию модули предоставляют
доступ (через файлы устройств) к своим подсистемам только суперпользователю.
Для обеспечения доступа к устройствам ADC7K всех пользователей, необходимо добавить
правила для подсистемы ядра \texttt{udev}, которые динамически изменяют права
доступа к файлам устройств при их регистрации в виртуальной файловой системе.
\begin{small}\begin{verbatim}
adc7k@adc7k ~/adc7k-linux $ sudo make install_udev_rules
\end{verbatim}\end{small}

\subsection{Настройка параметров ядра linux}

Для устройства ADC7K необходим достаточно большой, физически
непрерывный блок оперативной памяти, который невозможно выделить стандартными
средствами ядра linux в процесе работы, но возможно зарезервировать при старте ядра.
По умолчанию модуль \texttt{adc7k-cpci3u.ko} запрашивает память 16MiB начиная с
адреса 256M \texttt{(0x10000000-0x10ffffff)}. Для удовлетворения этого условия
мы должны отметить этот диапазон как зарезервированый и заблокировать его от использования
ядром linux, задав параметр ядра \texttt{memmap=16M$\backslash$\$256M}, где первое значение
указывает размер блока резервируемой памяти, а вторая начальный адрес.
Для установки данного параметра ищем в файле настроек загрузчика операционной системы
строку загрузки ядра и добавляем требуемый нами параметр в конец строки.
В примере строка из конфигурационного файла загрузчика GRUB
\\\\до редактирования
\begin{small}\begin{verbatim}
linux /vmlinuz-3.16.0-4-586 root=UUID=0da944c4-eb6c-4fe5-86d6-6aa6d3543db1
ro quiet splash
\end{verbatim}\end{small}
и после редактирования
\begin{small}\begin{verbatim}
linux /vmlinuz-3.16.0-4-586 root=UUID=0da944c4-eb6c-4fe5-86d6-6aa6d3543db1
ro quiet splash memmap=16M\$256M
\end{verbatim}\end{small}

\subsection{Завершение установки}

Выполнив все необходимые операции для установки модулей, мы можем перезагрузить
операционную систему и, в случае наличия плат ADC7K \mbox{CompactPCI-3U}, модули
загрузятся автоматически.

\section{Порядок работы}

В этом разделе мы рассмотрим порядок использования модулей ядра linux для обнаружения,
управления и получения данных с плат ADC7K \mbox{CompactPCI-3U}.

\subsection{Загрузка модулей}

Как было упомянуто в предыдущем разделе, после установки модулей и при наличии
в системе плат ADC7K \mbox{CompactPCI-3U}, загрузка модулей происходит автоматически.

Здесь мы рассмотрим порядок ручной загрузки модулей.

Для загрузки модулей в ядро linux необходимо выполнить команду \texttt{modprobe adc7k-cpci3u}.
В результате выполнения команды будет просмотрена карта зависимостей модулей в
которой будет обнаружена зависисмость модуля adc7k-cpci3u от adc7k и произойдет
загрузка модулей, сначала adc7k, а затем adc7k-cpci3u.

\subsection{Просмотр поддерживаемых устройств}

Модуль adc7k регистрирует в ядре подсистему которая обеспечивает регистрацию
устройств в виртуальной файловой системе. Доступ пользователей к этой подсистеме
осуществляется через файл устройства \texttt{/dev/adc7k/subsystem}. Для данного
файла реализована толь одна операция --- операция чтения, в результате которой
мы получаем объект в формате JSON. Данный объект содержит только один элемент ---
массив ''boards''. В этом массиве перечислены все платы, которые зарегистрированы
подсистемой adc7k. Поскольку формат JSON текстовый, мы можем вывести содержимое
файла подсистемы в терминале. Здесь и далее мы производим доступ к файлам
устройств права доступа которых расширены для всех пользователей.
Вывод подсистемы не содержащей плат ADC7K
\begin{small}\begin{verbatim}
adc7k@adc7k ~ $ cat /dev/adc7k/subsystem
{
    "boards": [
    ]
}
adc7k@adc7k ~ $
\end{verbatim}\end{small}
Зарегистрирована одна плата ADC7K \mbox{CompactPCI-3U}
\begin{small}\begin{verbatim}
adc7k@adc7k ~ $ cat /dev/adc7k/subsystem
{
    "boards": [
        {
            "driver": "adc7k_cpci3u",
            "path": "adc7k!board-cpci3u-03-09-0"
        }
    ]
}
adc7k@adc7k ~ $
\end{verbatim}\end{small}
Элемент массива ''boards'' также является JSON-объектом и, в свою очередь,
содержит два элемента ''driver'' и ''path''. Значение элемента ''driver''
содержит имя модуля, который облуживает данную плату. А в значении элемента
''path'' указан путь к файлу устройства управления и мониторинга платы. Как можно
заметить данный путь не является путем к файлу в стиле UNIX и для непосредственного
использования его необходимо модифицировать. В исходном значении пути необходимо
заменить все символы '\texttt{!}' на '\texttt{/}' и добавить префикс
\texttt{/dev/}. В результате, для значения пути из примера, получим
\texttt{/dev/adc7k/board-cpci3u-03-09-0}.

Тепер, когда мы знаем путь к файлу устройства интересуемой нас платы, можем
посмотреть информацию которую содержит файл данной платы. Выводим
содержимое файла в терминале
\begin{small}\begin{verbatim}
adc7k@adc7k ~ $ cat /dev/adc7k/board-cpci3u-03-09-0
{
    "type": "CompactPCI-3U",
    "sampler": {
        "rate": 50000000,
        "length": 1048576,
        "max_length": 1048576,
        "divider": 0
    },
    "channels": [
        {
            "name": "12",
            "path": "adc7k!board-cpci3u-03-09-0-channel-12",
            "transactions": 283912,
            "buffer": {
                "address": 268435456,
                "size": 4194304
            }
        },
        {
            "name": "34",
            "path": "adc7k!board-cpci3u-03-09-0-channel-34",
            "transactions": 283912,
            "buffer": {
                "address": 272629760,
                "size": 4194304
            }
        },
        {
            "name": "56",
            "path": "adc7k!board-cpci3u-03-09-0-channel-56",
            "transactions": 283912,
            "buffer": {
                "address": 276824064,
                "size": 4194304
            }
        },
        {
            "name": "78",
            "path": "adc7k!board-cpci3u-03-09-0-channel-78",
            "transactions": 283912,
            "buffer": {
                "address": 281018368,
                "size": 4194304
            }
        }
    ]
}
adc7k@adc7k ~ $
\end{verbatim}\end{small}
Прочитаный объект JSON содержит три элемента. Первый элемент строковый ''type''
носит информационный характер и содержит название типа платы. Второй элемент ---
объект ''sampler''. И третий элемент --- массив объектов ''channels''. В объекте
''sampler'' содержится информация о частоте дискретизации --- ''rate'', текущей и
максимальной длине выборки --- ''length'' и ''max\_length'' соответственно и
параметр прореживания данных ''divider''. В элементе массива ''channels'' содержится
объект с информацией о канале платы, точнее о двухканальной группе. Элемент ''name''
содержит имя канала, ''path'' --- путь к файлу устройства, ''transactions'' ---
количество DMA-транзакций данных канала и ''buffer'' --- объект содержащий адрес
(''address'') и размер (''size'') буфера DMA. Пути файлов устройств каналов для
практического применения также необходимо модифицировать как и путь файла устройства
платы.

\subsection{Управление платой ADC7K CompactPCI-3U}

Управление платой ADC7K \mbox{CompactPCI-3U} осуществляется записью команд в
файл устройства соответствующей платы. Команда представляет собой символьную строку.
За один цикл открытия/закрытия файла можно передать одну команду.
Далее приведен перечень доступных команд:
\begin{itemize}
\item[--] sampler.start(mode) - запуск процесса оцифровки и передачи данных.
где при mode=0 --- одиночный запуск, mode=1 --- непрерывный процесс.
\item[--] sampler.stop() - остановить непрерывный процесс оцифровки и передачи данных.
\item[--] sampler.length(value) - установить длину данных единичной выборки. где \(1 \leq value \leq max\_length\).
\item[--] sampler.divider(value) - установить параметр прореживания данных. где \(0 \leq value \leq 255\).
\item[--] board.reset() - сброс платы (при инициализации устройства подается автоматически).
\item[--] ddr.reset() - сброс контроллера DDR памяти (при инициализации устройства подается автоматически).
\item[--] adc.reset() - сброс АЦП (при инициализации устройства подается автоматически).
\item[--] adc.write(addr,data) - запись в регистры управления АЦП. где addr=0xXX --- адрес регистра, data=0xYYYY --- данные.
\end{itemize}

Перечисленные выше команды можно подавать как программно так и из терминала или запуская скрипт оболчки.
Пример сброса платы из командной строки
\begin{small}\begin{verbatim}
adc7k@adc7k ~ $ cat >> /dev/adc7k/board-cpci3u-03-09-0 << EOF
board.reset()
EOF
adc7k@adc7k ~ $
\end{verbatim}\end{small}

\subsection{Чтение данных}

Данные, получаемые в процесе оцифровки, можно прочитать из файла устройства канала.
Операции чтения работают как в блокирующем так и неблокирующем режиме.
Для этих файлов также доступны функции отображения данных буферов DMA в пространнстве
пользователя, что полностью устраняет копирование данных.
Одну из реализаций процедуры чтения данных (неблокирующий режим, отображение буфера DMA)
можно посмотреть в программе демонстраторе.

\end{document}
